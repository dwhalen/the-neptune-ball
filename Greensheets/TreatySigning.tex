\documentclass[green]{NeptuneBall}
\begin{document}
\name{\gTreaty{}}

As someone involved with the treaty negotiations, you know about the signing procedure for the treaty. A treaty exists in one of four states: Incomplete, Drafted, Ratified, and Vetoed. A treaty is considered Incomplete until it has a choice circled for every relevant choice, and asignature from the acting ruler of \pAtlantis{} and the currently highest ranking \pPacifica{{}n in game. At the start of game, this is \cKing{\King} \cKing{} and \cPrince{\Prince} \cPrince{} respectively. Once Complete, a treaty can be Vetoed by anyone with Veto power on the list below. Once Vetoed, a treaty can not be made valid again by any means, and a new one must be drawn up. 

\begin{enumerate}
\item The acting ruler of \pAtlantis{} (\cKing{\King} \cKing{} at the start of game)
\item The current highest ranking \pPacifica{}n (\cPrince{\Prince} \cPrince{} at the start of game)
\item \cPriest{}, representative from the Explorer's Guild.
\item \cGeneral{}, representative from the Guardian's Guild.
\item \cSlave{}, representative from the Merchant's Guild.
\end{enumerate}

A complete treaty is considered Ratified if it has signatures from every player involved in talks that currently holds veto power. or it has been publicly displayed for at least 15 minutes without Vetoes by the end of the game. If a member of the guilds is dead, their signature is not required to ratify a treaty and their veto is lost - however, if a representative for \pAtlantis{} or \pPacifica{} cannot be found, the treaty cannot be signed. A treaty, once ratified by the former method, cannot be made invalid - only physical destruction of that copy can make it invalid at that point.

In order for a signature on a new treaty to be valid, the previous version of the treaty must be first made invalid, either through a veto or by physically destroying it.

\end{document}