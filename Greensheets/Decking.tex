\documentclass[green]{NeptuneBall}
\begin{document}
\name{\gDecking{}}

\emph{(This greensheet details how to attempt to break into a lock protected by magic. (Assume all locks in game are magical unless you know otherwise).)}

Magical locks, while more secure than mundane locks, are still susceptible to being picked. Picking locks is a delicate and often time consuming process. The difficulty of picking a lock depends on the lock itself. If anyone asks what you are doing, you must tell them that you are fiddling with the lock in an obvious attempt to break in.\\ \\

How to Attempt to Pick a Lock:
\begin{enumerate}
	\item Shuffle the deck 7 times.
	\item Deal out {\bf 1} card. This is your {\em working hand}.
	\item Deal out 5 cards in a line below your working hand. This is your {\em dynamic library}.
	\item You may swap out {\bf one} card in your {\em working hand} with one card in your {\em dynamic library}.
	\item Discard all 5 cards in your {\em dynamic library}.
	\item Repeat steps 3-5 until you have fulfilled your success condition or you run through the teck.  If you run out of deck, you must start over from step 1.
\end{enumerate}

{\bf Success Condition by Lock Difficulty:}\\
\begin{tabular}{||r|l||}
\hline\hline
Lock Difficulty	& Required hand\\
\hline
0	& Straight of 4\\
1	& Straight of 5\\
2	& Straight of 6\\
3	& Straight of 7\\
4	& Straight of 8\\
\hline\hline 
\end{tabular}

\vspace{10 mm}

You may notice that several locks are impossible. This is intentional. If you attempt to pick a lock and fail, you may try again immediately (continue the session), or you may give up (end the session). If you have tried to pick a lock 2x in the same session, the third time you try to pick the lock in the same session, you may reduce the lock difficulty by 1.

There may be other ways to reduce the difficulty of a lock.

\end{document}