 %%%%
%%
%% This file sets up the Sign and Label datatypes and creates Sign and
%% Label macros.
%%
%% Signs generally represent interesting parts of game area, usually
%% as things posted on walls.  Labels represent other things, often on
%% or inside envelopes, that are part of complex mechanics.
%%
%% The default value for \MYloc will inherit location from the Place
%% or Sign most immediately up the ownership tree.  Override this by
%% setting \MYloc to anything (even blank).
%%
%% Sign is for full-sized signs that would cover most of a large
%% manila envelope; SignMedium is for signs sized to half-sized manila
%% envelopes; SignSmall is for signs sized for small manila envelopes
%% (the same size as item cards).  Label, LabelMedium, and LabelSmall
%% are analagous, but they don't have a \takedownby note at the
%% bottom.  You can always use a sign or label without an envelope or
%% with a differently-sized envelope.  Choose which based on
%% visibility and content.
%%
%% SignTiny is for signs you want to be hard to find; it is small and
%% does not have a \takedownby note.  SignDot is for a very small
%% "dot" which only has a title.
%%
%% SignStrip produces a strip of paper (without a \takedownby note)
%% with labels on the outside that show on both sides if you fold it
%% in half.  These are a convenient alternative to sub-envelopes. They
%% can also be used for "s-packets" taped to walls (see
%% Extras/README-s-packets).
%%
%% LabelCover produces a label similar to the cover to a research
%% notebook.  LabelPage, likewise, produces a page.
%%
%% EOG is for full-sized end-of-game signs.
%%
%%%%%

\DECLARESUBTYPE{Sign}{Element}
\PRESETS{Sign}{
  \FD\MYloc	{\mylocation} %% real-space location
  \FD\MYtext	{} %% text of sign
  }
\POSTSETS{Sign}{
  \edef\mylocation{\MYloc}
  \protected@edef\@ownerstring{%
    \MYname%
    \ifx\mylocation\empty\else\ (\mylocation)\fi%
    }
  }
\def\mylocation{}

\def\loc#1{\rs\MYloc{#1}}

\DECLARESUBTYPE{SignMedium}{Sign}
\DECLARESUBTYPE{SignSmall}{Sign}
\DECLARESUBTYPE{SignTiny}{Sign}
\DECLARESUBTYPE{SignDot}{Sign}
\PRESETS{SignDot}{\s\MYtext{}}

\DECLARESUBTYPE{Label}{Sign}
\PRESETS{Label}{\s\MYloc{}}
\DECLARESUBTYPE{LabelMedium}{Label}
\DECLARESUBTYPE{LabelSmall}{Label}

\DECLARESUBTYPE{SignStrip}{Sign}
\DECLARESUBTYPE{LabelCover}{Label}
\DECLARESUBTYPE{LabelPage}{Label}

\DECLARESUBTYPE{EOG}{Sign}
\PRESETS{EOG}{%
  \s\MYname	{End Of Game}
  \s\MYtext	{{\bf\Huge You may not pass through here.}}
  }


%%%%%
%% \signbig[<location>]{<name>}{<text>}
%% \eog[<location>]
%%
%% \signmdeium[<location>]{<name>}{<text>}
%% \signsmall[<location>]{<name>}{<text>}
%% \signtiny[<location>]{<name>}{<text>}
%% \signdot[<location>]{<name>}
%%
%% \labelbig{<name>}{<text>}
%% \labelmedium{<name>}{<text>}
%% \labelsmall{<name>}{<text>}
%%
%% \signstrip[<location>]{<name>}{<text>}
%% \labelcover{<name>}{<text>}
%% \labelpage{<name>}{<text>}
\newinstance{Sign}{\signbig[3][\mylocation]}{
  \s\MYloc{#1}\s\MYname{#2}\s\MYtext{#3}}
\newinstance{EOG}{\eog[1][\mylocation]}{\s\MYloc{#1}}

\newinstance{SignMedium}{\signmedium[3][\mylocation]}{
  \s\MYloc{#1}\s\MYname{#2}\s\MYtext{#3}}
\newinstance{SignSmall}{\signsmall[3][\mylocation]}{
  \s\MYloc{#1}\s\MYname{#2}\s\MYtext{#3}}
\newinstance{SignTiny}{\signtiny[3][\mylocation]}{
  \s\MYloc{#1}\s\MYname{#2}\s\MYtext{#3}}
\newinstance{SignDot}{\signdot[2][\mylocation]}{
  \s\MYloc{#1}\s\MYname{#2}}

\newinstance{Label}{\labelbig[2]}{
  \s\MYname{#1}\s\MYtext{#2}}
\newinstance{LabelMedium}{\labelmedium[2]}{
  \s\MYname{#1}\s\MYtext{#2}}
\newinstance{LabelSmall}{\labelsmall[2]}{
  \s\MYname{#1}\s\MYtext{#2}}

\newinstance{SignStrip}{\signstrip[3][\mylocation]}{
  \s\MYloc{#1}\s\MYname{#2}\s\MYtext{#3}}
\newinstance{LabelCover}{\labelcover[2]}{
  \s\MYname{#1}\s\MYtext{#2}}
\newinstance{LabelPage}{\labelpage[2]}{
  \s\MYname{#1}\s\MYtext{#2}}


%%%%%
%% \sEOG{}
%% use \sEOg[\loc{<location>}]{} for EOG sign at a specific place
\NEW{EOG}{\sEOG}{
  }


%%%%%%%%%%%%%%%%%%%%%%%%%%%%%%%%%%%%%%%%%%%%%%%%%%%%%%%%%%%%%%%%%%

\NEW{Sign}{\sTest}{
  \s\MYname	{A Room}
  \s\MYloc	{10-250}
  \s\MYtext	{A lecture hall with large, sliding blackboards.}
  }


%%%%%%%%%%%%%%%%%%%%%%%%%%%%%%%%%%%%%%%%%%%%%%%%%%%%%%%%%%%%%%%%%%

%% The hotel rooms
%\NEW{Sign}{\sHotelRooms}{
%  \s\MYname	{Hotel Room Access}
%  \s\MYloc	{- 004}
%  \s\MYtext	{There are three ways to gain entrance to a hotel room:
%
%{\bf 1. } Possess the appropriate room key.
%
%{\bf 2. } You may hit the door with a {\bf CR of 5} or greater.
%
%{\bf 3. } Hack your way in. The lock has a difficulty level of {\bf 1} unless the room belongs to, in which case the lock has a difficulty level of {\bf 2}.
%
%Once you have completed one of the above requirements, you may stand with your hand on the room sign for 30 seconds to enter the room. You may then freely stash items that are stashable, and take items that are in the room.}
%}
 
%Library
\NEW{Sign}{\sMuseum}{
  \s\MYname	{The Royal Museum}
  \s\MYloc	{- 002}
  \s\MYtext	{This area houses the royal treasury and the royal archives. It is a large, open space with vaulted ceilings and many priceless artifacts scattered among the ancient texts.}
  \s\MYitems {}
} 

\NEW{Sign}{\sMuseumItems}{
  \s\MYname	{The Royal Treasury}
  \s\MYloc	{- 002}
  \s\MYtext	{There are many valuable items in the Royal Treasury. They chronicle the history of \pAtlantis{}n art. They are on display for all citizens to enjoy.
  
  The acting ruler of \pAtlantis{} has the authority to dispense these items as rare and valuable gifts. This only happens once in a lifetime or so, since the items are so dear to \pAtlantis{}. It is a sign of highest honor to receive one.
  
  Some items may also be vulnerable to being broken or destroyed. It is a crime punishable by 10 years of service to destroy an artifact housed in the Royal Treasury.}
  \s\MYitems {}
} 

\NEW{Sign}{\sArtifactOne}{
  \s\MYname	{Some Fancy Artifact}
  \s\MYloc	{- 002}
  \s\MYtext	{This Fancy Artifact is a national treasure of \pAtlantis{}. It is very valuable.}
  \s\MYitems {}
} 

\NEW{Sign}{\sArtifactTwo}{
  \s\MYname	{Necklace of Diana} %%Diana is the greek goddess of the moon
  \s\MYloc	{- 002}
  \s\MYtext	{This piece is an elaborate necklace. It is the crowning achievement of a famous \pAtlantis{}n jeweler, \cJeweler{}. It's centerpiece is a huge black pearl.
  
  \emph{You can hit this piece with a CR 6+ attack to break it. If you do, take \iPearl{} from the envelope below and swap this sign with the one under it.}
  }
  \s\MYitems {\iPearl{}}
} 

\NEW{Sign}{\sArtifactTwoBroken}{
  \s\MYname	{Ruined Necklace of Diana}
  \s\MYloc	{- 002}
  \s\MYtext	{This piece was once an elaborate and expensive necklace. The protective case has been smashed and the necklace has been mangled. Whatever used to hang as the pendant has been removed.}
  \s\MYitems {}
} 

\NEW{Sign}{\sArtifactThree}{
  \s\MYname	{Cassandra's Mirror} %%
  \s\MYloc	{- 002}
  \s\MYtext	{This is the pedestal display for \iScryingMirror{}.
  
   \emph{You can hit this display with a CR 6+ attack to break the glass protecting the mirror. If you do, take \iScryingMirror{} from the envelope below and swap this sign with the one under it.}}

  \s\MYitems {\iScryingMirror{}}
} 

\NEW{Sign}{\sArtifactThreeBroken}{
  \s\MYname	{Shattered Pedestal}
  \s\MYloc	{- 002}
  \s\MYtext	{This shattered pedestal once held \iScryingMirror{}. It has been destroyed and the treasure stolen.}
  \s\MYitems {}
} 

\NEW{Sign}{\sArtifactFour}{
  \s\MYname	{Ornate Box} 
  \s\MYloc	{- 002}
  \s\MYtext	{This is the pedestal display for an ornate box inscribed with ancient runes.
  
  A small plaque under the box explains that the runes on the box are believed to be \pPacifica{}n in origin.
  
  There is a circular indentation on the front of the box, bearing the \pPacifica{}n royal crest. If you have a signet ring bearing the crest of \pPacifica{}, you may open the box. If you do so, replace this sign with the one under it.}

  \s\MYitems {\iJournal{}}
} 

\NEW{Sign}{\sArtifactFourOpened}{
  \s\MYname	{Opened Box}
  \s\MYloc	{- 002}
  \s\MYtext	{The ornate box stands open. There is a cavity inside big enough to hold a small book.
  
  A small plaque under the box explains that the runes on the box are believed to be \pPacifica{}n in origin.
  
  There is a circular indentation on the front of the box, bearing the \pPacifica{}n royal crest.}
  \s\MYitems {}
} 

\NEW{Sign}{\sBookshelf}{
  \s\MYname	{A Bookshelf}
  \s\MYloc	{- 002}
  \s\MYtext	{This is a bookshelf. There are many books here for perusal, both casual and intense.}
  \s\MYitems {}
}

\NEW{Sign}{\sBookshelfRapsheet}{
  \s\MYname	{A Bookshelf}
  \s\MYloc	{- 002}
  \s\MYtext	{This is a bookshelf. There are many books here for perusal, both casual and intense.
  
    {\bf You may only take an item from this bookshelf if your $\gamma$ score is 2.}
  
  If you would like to take a book from the bookshelf, place both hands on this sign for 30 seconds.}
  \s\MYitems {\iRapSheet{}}
} 

\NEW{Sign}{\sBookshelfBook}{
  \s\MYname	{A Bookshelf}
  \s\MYloc	{- 002}
  \s\MYtext	{This is a bookshelf. There are many books here for perusal, both casual and intense.
  
  {\bf You may only take an item from this bookshelf if your $\alpha$ score is 2.}
  
  If you would like to take a book from the bookshelf, place both hands on this sign for 30 seconds.}
  \s\MYitems {\iBook{}}
}  

\NEW{Sign}{\sBlackboard}{
  \s\MYname	{Blackboard}
  \s\MYloc	{-002}
  \s\MYtext	{A blackboard with a few scribbles on it. Nothing too interesting.
  
  {\bf You may only interact with this sign if you have an $\alpha$ score of 2.}
  
  If you would like to take an item from this sign, place both hands on it for 30 seconds.
  \s\MYitems {\multi{5}{\iChalk{}}}
  }
}

\NEW{Sign}{\sInheritance}{
  \s\MYname	{Book of Lineage}
  \s\MYloc	{-002}
  \s\MYtext	{A book documenting the long, proud lineage of \pAtlantis{}n rulers. It also contains a document detailing the complicated inheritance for the crown of \pAtlantis{}. If you would like to view this inheritance, you may lift this sign and look at the one under it.}
  \s\MYitems {}
  }

\NEW{Sign}{\sLineageOne}{
  \s\MYname	{Atlantican Inheritance Pg 1}
  \s\MYloc	{-002}
  \s\MYtext	{
  It may come to pass that the currently confirmed or acting ruler of \pAtlantis{} dies or is otherwise found unfit to rule. This can come about in a number of ways. They could be found mentally unstable, demonstrated to have usurped power from the rightful heir, be an illicit form of magical creature (such as a shapeshifter), or be proven to have committed treason. The supreme justice, judge of the highest court of the land may be summoned (find a GM or NPC) to verify any claim that the ruler is unfit. If any of these conditions can be proven to have come to pass, the monarch is then immediately removed. 

To determine who is next in line of succession, follow the list below until you find the first eligible candidate. Assume all classifications that could have multiple members (more than one male child for example) process in order of age. Women who marry into another line of succession forfeit their position in line for the throne of their old kingdom. Bastards of proven lineage are eligible. For example, a bastard cousin is considered in the line of succession as appropriate.

From the last \emph{confirmed} (not acting) ruler of \pAtlantis{} power passes to:
\begin{multicols}{2}
\begin{enumerate}
\item Male children
\item Female children
\item Husband or Wife
\item Brothers 
\item Sisters
\item Brothers-in-law
\item Sisters-in-law
\item The Father
\item The Mother
\item Uncles on the {\bf Father�s} side
\item Uncles on the {\bf Mother�s} side
\item Aunts on the {\bf Father�s} side
\item Aunts on the {\bf Mother�s} side
\item Male cousins on the {\bf Father�s} side
\item Male cousins on the {\bf Mother�s} side
\item Female cousins on the {\bf Father�s} side
\item Female cousins on the {\bf Mother�s} side
\end{enumerate}
\end{multicols}

If for some reason, no eligible candidate can be  found, the Council of Advisors will elect a new monarch in a closed meeting as soon as possible. (out of the scope of the game)

Cont. on PG 2 (You may lift this sign up and read the sign under it.)
  }
  \s\MYitems {}
}

\NEW{Sign}{\sLineageTwo}{
  \s\MYname	{Atlantican Inheritance Pg 2}
  \s\MYloc	{-002}
  \s\MYtext	{

If power passes to a new person, the acting monarch is considered only a steward until confirmed. If the discredited person manages to reestablish their eligibility before the acting monarch is confirmed they can re-assume power immediately. The process of actually confirming the monarch takes about a week and is the duty of the Council of Advisors.

In order to pass power to the acting monarch:
\begin{enumerate}
\item The supreme justice must preside over the ceremony
\item The eligibility of the candidate must be verified by at least 2 \pAtlantis{}ns (using the chart above).
\item They need to take an oath to protect \pAtlantis{} in front of 3 witnesses (at least 1 must be \pAtlantis{}n).
\end{enumerate}

Following the completion of the ceremony, NPC pages will inform everyone in game of the change.

\iTrident{} rightfully belongs to the acting ruler of \pAtlantis{}. The acting monarch should not hesitate to use his or her extensive authority to acquire  \iTrident{} if anyone is foolish enough to withhold it.

  }
  \s\MYitems {}
}
%Dungeon
\NEW{Sign}{\sDungeons}{
  \s\MYname	{The Dungeons}
  \s\MYloc	{Hallway off - 002}
  \s\MYtext	{These are the royal dungeons. Party-goers probably shouldn't be in here. You may not enter unless you are incarcerated, and only 3 people may be incarcerated at a time.}
  \s\MYitems {}
}

\NEW{Sign}{\sBarnacles}{
  \s\MYname	{Barnacles}
  \s\MYloc	{Hallway off - 002}
  \s\MYtext	{There is a growth of barnacles on the wall here. The servants have neglected to clear it off. 
  
  {\bf You may spend 30 seconds scraping one off of the wall. If you do, take one item from the envelope below.}}
  \s\MYitems {\multi{5}{\iBarnacle{}}}
} 

%%Stairwell
%\NEW{Sign}{\sOldMan}{
%  \s\MYname	{The Palace Storyteller}
%  \s\MYloc	{Stairwell}
%  \s\MYtext	{There is an old merman floating here. {\bf He accosts all passersby with epic tales of powerful magical artifacts and impossible wishes.} No one seems surprised he is here, but most people aren't paying much attention to him. He seems sad but earnest.
%  
%  {\bf If you wish to listen to one of the old man's stories, you may lift this sign up and read what is on the sign below.}}
%} 

%\NEW{Sign}{\sOldManStory}{
%  \s\MYname	{The Story of the Wishing Stone}
%  \s\MYloc	{Stairwell}
%  \s\MYtext	{The old merman is happy to have an audience. He spins a melancholy tale:
%  \emph{Come closer young one, and I will tell you the story of \iWishingStone{\MYname}. This innocuous little stone has the power to grant wishes -- even the most impossible ones. There was once an old merman who had seen great sorrow in his life. He had lost his wife to \pPolio{}, and his children to the war. He was alone in his old age. He wished more than anything to see his family again.
%  
%  The old man therefore set out on a quest to find \iWishingStone{\MYname}. This stone, the old man knew, could allow him to see his family again.  He journeyed for many days and nights, asking at every house he encountered, but none could -- or would -- help him. They said ``Go away old man. Wishes are for the young.''
%  
%  Eventually the old man found his way to the Angola plain, where he knocked on the door of a young mermaid's house, to ask if she knew of \iWishingStone{\MYname}. The mermaid was sad and quiet, but she invited him in for a meal. The merman asked her why one so young and beautiful was sad. She explained that her dear brother had gone on a quest for \iWishingStone{\MYname}. He found it, but before he could make his wish, he was killed by another, jealous of her brother's success.
%  
%  The old merman was saddened to hear that even the young suffer such pain. He looked back at his own life, and thought that he had many good parts, and all was not dark. He therefore offered to wish for happiness in the mermaid's future, instead of seeing his own family again.
%  
%  The mermaid smiled then. She got up, and went to a chest in the corner of her house. She drew from within it, a humble stone. She brought it back to the old man, and said ``You are worthy.'' She then melted away into sea foam, and was carried off on the current.}
%  
%  As the storyteller's tale concludes, his eyes glitter mysteriously. ``Would you like to search for the \iWishingStone{\MYname}? Do you think you are worthy of it?''
%  
%  {\bf If you would like to search for \iWishingStone{}, take a greensheet from the packet below and follow the instructions}
%  }
%  \s\MYgreens {\multi{10}{\gWS{}}}
%} 


%Banquet hall
\NEW{Sign}{\sBanquetHall}{
  \s\MYname	{The Banquet Hall}
  \s\MYloc	{-013}
  \s\MYtext	{This is the royal banquet hall. No expense was spared in decorating this room with rare shells from across \pAtlantis{}. The banquet will take place here at \cTTwo{}.}
  \s\MYitems {}
}

\NEW{Sign}{\sStage}{
  \s\MYname	{The Band Stage}
  \s\MYloc	{-013}
  \s\MYtext	{This is the stage for the band. There are a number of instruments lying around here -- It looks like the musicians each brought several so they could play many kinds of music.
  
  If you would like to borrow an instrument, put 1 hand on this sign for 30 seconds, then take an instrument at random from the envelope.
  
  {\bf You must return the instrument within 10 minutes otherwise the band will be very upset.}}
  \s\MYitems {\iDrums{} \iSaxaphone{} \iGuitar{} \iBass{}}
}

\NEW{Sign}{\sRevealed}{
  \s\MYname	{S Packet}
  \s\MYloc	{- 013}
  \s\MYtext	{You cannot interact with this sign unless you know otherwise.}
  \s\MYabils {\aRevealed{}}
	\s\MYitems {\iShapeshifterSpell{}}
}

%Explorer's guild
\NEW{Sign}{\sExplorersDoor}{
  \s\MYname	{The Explorer's Guild}
  \s\MYloc	{Hallway near -015}
  \s\MYtext	{This is an office of the Explorer's Guild in the palace. If you are an Explorer's Guild member, the door recognizes you and lets you pass freely (You may look at the sign under this one immediately.) 
  
  If you are not an Explorer's Guild member and would like to break in, you may try to beat the lock by decking a hand of 4. (This is a magical lock).}
  \s\MYitems {}
}

\NEW{Sign}{\sExplorersInventory}{
  \s\MYname	{The Explorer's Guild Inventory}
  \s\MYloc	{Hallway near -015}
  \s\MYtext	{This is the inventory of the Explorer's Guild office. You may take any {\bf 1} of the following items by searching through the inventory for a period of time, then crossing it off the list below. If you are an Explorer's Guild member, this action takes 1 minute as the magic of the room aids you. Otherwise, it takes 2 minutes. You may as many items as you like, after fulfilling the time requirement for it.
  \begin{multicols}{2}
  \begin{enumerate}
	\item \iAnglerFish{}
	\item \iAnglerFish{}
	\item \iAnglerFish{}
	\item \iAnglerFish{}
	\item \iAnglerFish{}
	\item \iAnglerFish{}
	\item \iManOfWar{}  (Dangerous)
	\item \iManOfWar{}  (Dangerous)
	\item \iManOfWar{}  (Dangerous)
	\item \iManOfWar{}  (Dangerous)
	\item \iDeepClam{}
	\item \iDeepClam{}
	\item \iElectricEel{}
	\item \iElectricEel{}
	\item \iElectricEel{}
	\item \iIceFish{}
  \item \iIceFish{}
	\item \iSquid{}
	\item \iSquid{}
	\item \iSquid{}
	\item \iSquid{}
  \end{enumerate}
  \end{multicols}
  If you take one of these items, cross it off of the list above. It is immediate apparent to anyone in this room what has been taken.
  }
  \s\MYitems {\iIceFish{} \iIceFish{} \multi{4}{\iSquid{}} \multi{6}{\iAnglerFish{}} \iElectricEel{} \iElectricEel{} \iDeepClam{} \iDeepClam{} \multi{4}{\iManOfWar{}}}
}

\NEW{Sign}{\sPileOfMaps}{
  \s\MYname	{A Big, Disorderly Pile of Maps}
  \s\MYloc	{Hallway near -015}
  \s\MYtext	{There is a big pile of maps in the corner here. They have clearly been thrown there without much care to their order. 
  
  {\bf You may not interact with this sign unless you know otherwise}}
  \s\MYitems {\iMapOfIndia{} \iMapOfAtlantica{} \iMapOfPacifica{}}
}

\NEW{Sign}{\sJournals}{
  \s\MYname	{A Bookshelf Full of Old Journals}
  \s\MYloc	{Hallway near -015}
  \s\MYtext	{There is a bookshelf overflowing with old journals here
  
  {\bf You may not interact with this sign unless you know otherwise}}
  \s\MYitems {\iNorthSeasJournal{}}
}


%caves
\NEW{Sign}{\sCaves}{
  \s\MYname	{The Caves}
  \s\MYloc	{-015}
  \s\MYtext	{This is a set of caves that are on the palace grounds but not part of the palace themselves. Party-goers should have no business here.}
  \s\MYitems {}
}

\NEW{Sign}{\sSnails}{
  \s\MYname	{The Caves}
  \s\MYloc	{-015}
  \s\MYtext	{This is a set of caves that are on the palace grounds but not part of the palace themselves. Party-goers should have no business here.}
  \s\MYitems {}
}

\NEW{Sign}{\sClamBed}{
  \s\MYname	{A bed of clams}
  \s\MYloc	{-015}
  \s\MYtext	{There are a bed of clams here. {\bf You may spend 1 minute to  break open one of the clams.} To do so, take one of the clams from the envelope below and open it after {\bf 1 minute} has passed.}
  \s\MYitems {\multi{9}{\iClam{}} \iPearlClam{}}
}

\NEW{SignMedium}{\sPearlStorage}{
  \s\MYname	{A Black Pearl}
  \s\MYloc	{-015}
  \s\MYtext	{You may only take the item from the $\alpha$ packet below if you have been directed to do so by a mechanic. If you do so, remove this sign and accompanying envelope.}
  \s\MYitems {\iPearl{}}
}

\NEW{Sign}{\sCrest}{
  \s\MYname	{The Magician's Guild Crest.}
  \s\MYloc	{-015}
  \s\MYtext	{The Crest of the now dissolved Magician's Guild has been hung on the cave wall here.}
  \s\MYitems {}
}

\NEW{Sign}{\sRunicCircle}{
  \s\MYname	{Magical Runic Circle}
  \s\MYloc	{-015}
  \s\MYtext	{A runic circle is inscribed on the ground here.}
  \s\MYitems {}
  \s\MYgreens{\gPotions{} \gPotions{} \gPotions{}}
  \s\MYabils{\aLesserDispel{} \aLesserDispel{} \aLesserDispel{}}
}

%Residential Wing
\NEW{Sign}{\sResidences}{
  \s\MYname	{The Residential Wing of the Palace}
  \s\MYloc	{-030}
  \s\MYtext	{This is the Residential Wing of the palace. The suite for the royal family, and many of their long term guests are found in this area.}
  \s\MYitems {}
}


%Garden
\NEW{Sign}{\sGardens}{
  \s\MYname	{The Palace Gardens}
  \s\MYloc	{-032}
  \s\MYtext	{These are the royal gardens. They are some of the most beautiful in all of \pAtlantis{}. It is a nice quiet place to talk, or take a stroll.}
  \s\MYitems {}
}

\NEW{Sign}{\sSeaFans}{
  \s\MYname	{A Rocky Outcropping}
  \s\MYloc	{-032}
  \s\MYtext	{There are many beautiful sea fans, coral and other ocean flora and fauna on this rock. It has clearly been carefully manicured for the palace.
  
  If you would like to take something from the garden, you may spend {\bf 30 seconds} with one hand the sign, and then take an item of your choosing.
  
  (The items available at this sign are essentially unlimited. If the envelope is empty, tell a GM.)}
  \s\MYitems {\multi{20}{\iSeaFan{}}}
}


%Pacifican Embassy
\NEW{Sign}{\sEmbassy}{
  \s\MYname	{The Pacifican Embassy}
  \s\MYloc	{-004}
  \s\MYtext	{This is the suite dedicated to the Pacifican delegates. It has been set up in some haste, but there are plans to establish a permanent embassy once the treaty is signed.}
  \s\MYitems {}
}

%Kitchen
\NEW{Sign}{\sKitchen}{
  \s\MYname	{The Palace Kitchens}
  \s\MYloc	{Across -004}
  \s\MYtext	{These are the royal kitchens. Everything is in chaos due to the banquet.}
  \s\MYitems {}
}

\NEW{Sign}{\sSeaSnake}{
  \s\MYname	{A jar of Sea Snake Tails}
  \s\MYloc	{Across -004}
  \s\MYtext	{There is a jar of snake tails on the counter here. They are a handy snack on the go, although some merfolk don't like the aftertaste.
  
  If you would like to take \iSeaSnake{}, you may spend {\bf 30 seconds} with one hand the sign, and then take one.
  
  (The items available at this sign are essentially unlimited. If the envelope is empty, tell a GM.)}
  \s\MYitems {\multi{20}{\iSeaSnake{}}}
}

\NEW{Sign}{\sSwordfishMeat}{
  \s\MYname	{A Meat Locker}
  \s\MYloc	{Across -004}
  \s\MYtext	{There is a Meat Locker here. It is used for keeping fish fillets cold.
  
  If you would like to take something from the locker, you may spend {\bf 1 minute} with one hand the sign, and then take an item of your choosing.
  }
  \s\MYitems {\multi{5}{\iSwordfish{}}}
}


%Cauldrons
\NEW{Sign}{\sEmptyCauldron}{
  \s\MYname	{An Empty Cauldron}
  \s\MYloc	{}
  \s\MYtext	{There is an empty cauldron here.
  
  {\bf You may not interact with this sign unless you know otherwise.}}
  \s\MYitems {}
}

\NEW{Sign}{\sFullCauldron}{
  \s\MYname	{A Bubbling Cauldron}
  \s\MYloc	{}
  \s\MYtext	{There is a bubbling cauldron here.
  
  {\bf You may not interact with this sign unless you know otherwise.}
  
  The color of the liquid inside is:
  
  }
  \s\MYitems {\multi{8}{\iHealingPotion{}} \multi{8}{\iPoison{}} \multi{5}{\iTransformationPotion{}} \multi{2}{\iDeadlyPoison} \multi{8}{\iBuffPotion} \multi{1}{\iShapeshifterPotion{}} \multi{1}{\iLovePotion{}}}
}

%%Music Box locations

\NEW{SignSmall}{\sPacketA}{
  \s\MYname	{Packet A}
  \s\MYloc	{ROOM NUMBER}
  \s\MYtext	{}
  \s\MYitems {}
}

\NEW{SignSmall}{\sPacketB}{
  \s\MYname	{Packet B}
  \s\MYloc	{ROOM NUMBER}
  \s\MYtext	{}
  \s\MYitems {}
}

\NEW{SignSmall}{\sPacketC}{
  \s\MYname	{Packet C}
  \s\MYloc	{ROOM NUMBER}
  \s\MYtext	{}
  \s\MYitems {}
}

\NEW{SignSmall}{\sPacketD}{
  \s\MYname	{Packet D}
  \s\MYloc	{ROOM NUMBER}
  \s\MYtext	{}
  \s\MYitems {}
}



%GM / oog signs