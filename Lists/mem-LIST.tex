%%%%%
%%
%% This file sets up the Mem, MemFold, and MemEnvelope datatypes, and
%% creates possible macros for each.
%%
%% The Mem datatype isn't really used directly; it's there so the
%% other datatypes can inherit and share its code.
%%
%%%%%

\DECLARESUBTYPE{Mem}{Element}
\PRESETS{Mem}{
  %% \MYname is the trigger
  \F\MYtext	%%  text
  }


%%%%%
%% MemFold and MemEnvelope are both subtypes of Mem.  MemFold is for
%% fold-n-staple style mempackets, MemEnvelope is for stuff-n-seal
%% style mempackets.  If you want a mempacket to contain interesting
%% contents, like sheets, abilities, and other mempackets, use a
%% MemEnvelope.
\DECLARESUBTYPE{MemFold}{Mem}
\DECLARESUBTYPE{MemEnvelope}{Mem}


%%%%%
%% MemCover and MemPage are for the cover and pages of mempacket
%% booklets, which resemble research notebooks.  These are good
%% substitutes for large piles of MemFolds, and can be useful for
%% things like amnesiac characters.
%%
%% Like MemFolds, MemPages shouldn't directly own any other elements
%% as contents.  Use MemEnvelope instead.
%%
%% MemPages are usually assigned to a MemCover (via \MYmems), with the
%% MemCover representing the entire booklet and assigned to a
%% character.
%%
%% A MemCover is not a mempacket in and of itself; its name is not its
%% trigger and its text is not a memory.
\DECLARESUBTYPE{MemCover}{Mem}
\PRESETS{MemCover}{
  \sd\MYtext	{Each page is a memory/event packet with a separate
		trigger.}
  }

\DECLARESUBTYPE{MemPage}{Mem}


%%%%%
%% \memfold{<trigger>}{<text>}
%% \memenvelope{<trigger>}{<text>}
%% \memcover{<name>}{<pages>}
%% \mempage{<trigger>}{<text>}
%% \startmembook{<name>} <pages> \endmembook
%%
%% These are wrappers around \INSTANCE, useful as 1-shots.
%% \startmembook...\endmembook is a simple wrapper around \memcover
%% that may have better syntax for use within character sheets (inside
%% a \starttag{mems}...\endtag block).
\newinstance{MemFold}{\memfold[2]}{
  \s\MYname{#1}\s\MYtext{#2}}
\newinstance{MemEnvelope}{\memenvelope[2]}{
  \s\MYname{#1}\s\MYtext{#2}}
\newinstance{MemCover}{\memcover[2]}{
  \s\MYname{#1}\s\MYmems{#2}}
\newinstance{MemPage}{\mempage[2]}{
  \s\MYname{#1}\s\MYtext{#2}}

\long\def\startmembook#1#2\endmembook{\memcover{#1}{#2}}


%%%%%%%%%%%%%%%%%%%%%%%%%%%%%%%%%%%%%%%%%%%%%%%%%%%%%%%%%%%%%%%%%%

\NEW{MemFold}{\mTest}{
  \s\MYname	{Test Trigger}
  \s\MYtext	{This is a Test of a fold-n-staple memory packet}
  }

  
%%%%%%%%%%%%%%%%%%%%%%%%%%%%%%%%%%%%%%%%%%%%%%%%%%%%%%%%%%%%%%%%%%


%%Potions
\NEW{MemFold}{\mLatentPoisonEffect}{
  \s\MYname	{Open immediately once you return to game.}
  \s\MYtext	{You have been poisoned! You are slightly nauseous, now mute, and will die in 5 minutes if you do not receive medical attention or consume a healing potion. If you do either of these things, you are cured-- you won't die and are no longer mute.}
  }

\NEW{MemFold}{\mDeadlyPoisonEffect}{
  \s\MYname	{Open immediately once you return to game.}
  \s\MYtext	{You have been poisoned! The poison is very powerful and will kill you in 10 minutes. Since you are moderately nauseous, your CR drops to zero (and cannot be increased) and you must walk heel to toe. Only an exceptionally powerful magical artifact (\iWishingStone{\MYnumber}) can save you.}
  }
